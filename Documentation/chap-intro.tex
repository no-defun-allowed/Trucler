\chapter{Introduction}
\pagenumbering{arabic}%

In section 8.5 of the second edition of the book ``Common Lisp, the
Language'' (also known as CLtL2) by Guy Steele
\cite{Steele:1990:CLL:95411}, a protocol for accessing compile-time
environments is defined.  That protocol has two main problems:

\begin{enumerate}
\item It is incomplete.  It does not provide for a way to query the
  environment for description about blocks or tags.
\item It is not extensible.  In order for an implementation to make
  one of the query functions return more information, additional
  return values would have to be defined.  However, such a change is
  not considered backward compatible, so this kind of extension is not
  recommended.
\end{enumerate}

\sysname{} introduces a protocol that solves these problems as
follows:

\begin{enumerate}
\item It contains additional query and augmentation functions for
  blocks and tags.
\item Instead of returning multiple values, the query functions return
  standard objects.  Accessors specialized to the classes of those
  objects provide the information that the protocol in CLtL2
  provides as multiple values.
\end{enumerate}

The term \emph{language processor} is used in this document to mean a
program that processes source code, such as a compiler or some other
code walker, with the intent of either modifying the source code, or
of generating some different representation for it.

In addition to providing a mechanism that solves the problems of the
protocol presented in CLtL2, we also add several new features that a
language processor must use to process source code.

The term \emph{client code} is used in this document to mean two
things:

\begin{enumerate}
\item Code that is specific to a \commonlisp{} implementation and that
  provides specialized code for one or more generic functions defined
  in this document.  Typically, such code will depend on the precise
  representation of lexical environments used by the implementation.
\item Code that is specific to the language processor that uses
  \sysname{} to obtain information about program elements in the
  source code that it processes.
\end{enumerate}

Both these types of specialization are introduced by means of a
parameter to \sysname{} functions called \texttt{client}.  Typically,
a \commonlisp{} implementation will provide a class such that an
instance represents the implementation, and also provide methods on
\sysname{} functions, specialized to this class.  A language processor
that needs further specialization can then define a subclass of this
class, so that these methods can be overridden or extended.
